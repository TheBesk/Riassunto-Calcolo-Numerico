\section{Introduzione}
Questo è un breve riassunto che ho scritto dopo aver fallito per innumerevoli volte la prova scritta di calcolo numerico. Per ogni domanda non riporterò la dimostrazione esatta che bisogna scrivere all'esame, ma concetti che aiutano a capire il senso (e quindi a memorizzare gli argomenti).\\
Prima di leggere questo documento consiglio di rivedere la teoria di analisi, in particolare gli argomenti che vengono trattati anche nel corso di calcolo.\\
Consiglio di leggere quanto scritto nelle prossime pagine con occhio critico: sicuramente saranno presenti molti errori, spero però che questi appunti siano utili per comprendere come studiare le varie dimostrazioni.\\
In seguito riporto alcune cose utili, che saranno ricorrenti nelle varie dimostrazioni:
\begin{itemize}
    \item La lettera greca $\xi$ il prof la utilizza (solitamente quando parla dei metodi iterativi per risolvere le equazioni) per indicare la \underline{vera} soluzione dell'equazione. Invece la successione ${x_n}$ indica la soluzione che ho trovato all'$n$-esimo passaggio
    \item Con stima si intende dare un valore ad una quantità che non conosco, confrontandola con delle quantità note. Quando si parla di stima si avranno delle disuguaglianze, e qui i teoremi che vengono utilizzati di solito sono il teorema dei carabinieri, disuguaglianza triangolare e altre disuguaglianze fondamentali (queste compaiono solo nella domanda sulle stime di condizionamento)
    \item stimare "da sotto" vuol dire (letteralmente), in una frazione, considerare solo il denominatore. Quando, in seguito, si riuniscono numeratore e denominatore, è importante cambiare il verso della disequazione (vedi ad es. la domanda sulla precisione di macchina).
\end{itemize}